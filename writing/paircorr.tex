\documentclass[11pt,letterpaper]{article}
\usepackage[top=1in,bottom=1in,left=1in,right=1in]{geometry}
\usepackage{natbib}      % http://merkel.zoneo.net/Latex/natbib.php
\usepackage{palatino}
\bibpunct{(}{)}{;}{a}{,}{,}
\usepackage{chngpage}
\usepackage{stmaryrd}
\usepackage{amssymb}
\usepackage{amsmath}
\usepackage{amsthm}
\usepackage{graphicx}
\usepackage{lscape}
\usepackage{subfigure}
\usepackage[usenames,dvipsnames]{color}
\definecolor{myblue}{rgb}{0,0.1,0.6}
\definecolor{mygreen}{rgb}{0,0.3,0.1}
\usepackage[colorlinks=true,linkcolor=black,citecolor=mygreen,urlcolor=myblue]{hyperref}
\newcommand{\bocomment}[1]{\textcolor{Bittersweet}{[#1 -BTO]}}
\newenvironment{itemizesquish}{\begin{list}{\labelitemi}{\setlength{\itemsep}{0em}\setlength{\labelwidth}{2em}\setlength{\leftmargin}{\labelwidth}\addtolength{\leftmargin}{\labelsep}}}{\end{list}}
\newcommand{\ignore}[1]{}
\newcommand{\transpose}{^\mathsf{T}}
\newcommand{\inner}[1]{\langle #1 \rangle}
\newcommand{\smallsec}[1]{\noindent \textbf{#1\ }}

\newcommand{\solution}[1]{{\color{Blue}[\textbf{Solution:} #1]}}
% \newcommand{\solution}[1]{}
\theoremstyle{definition}
\newtheorem{question}{Question}[section]
% \newtheorem{question}{Question}

\title{
Formal properties of word pair correlations
}

\author{
}

%\date{December 9, 2009}

\begin{document}
\maketitle

\section{Definition}

The ``word pair correlation function'' can be defined for a single wordtype, or across all words.

Across all words, define it as the probability a pair of tokens, spaced $r$ distance apart, are the same word.  So if you picked a position $t$ at random, what's the chance the same word is at $t+r$?  This is:

\[ g(r) = p(w_t = w_{t+r}) \]

Consider the case $r=1$.  This can be rewritten in a history conditional form as follows.

\begin{align}
g(r) &= p(w_t=w_{t+1}) \\
&= \sum_v p(w_t=v,\ w_{t+1}=v)  \label{e:introduce-v} \\
&= \sum_v p(w_{t+1}=v \mid w_t=v)\ p(w_t=v) \label{e:condprob}
\end{align}

Step \ref{e:introduce-v} decomposes the probability into the sum over all words that are possible.  Step \ref{e:condprob} is definition of conditional probability.

To do:
\begin{enumerate}
\item Derive $g(r)$ under a zeroth-order (independent unigrams) assumption.
\item Derive $g(r=1)$ under a first-order (bigram model) assumption.
\item Derive $g(r)$ under a first-order model.
\item Derive $g(r)$ under an arbitrary order model .. or maybe just second order, that sounds easier.
\end{enumerate}


% \bibliographystyle{plainnat}
% \bibliography{brenocon}
\end{document}
